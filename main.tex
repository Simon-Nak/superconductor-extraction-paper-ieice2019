\documentclass{article}
\usepackage[utf8]{inputenc}
\usepackage{authblk}
\usepackage{tabularx}

\title{Proposal of Automatic Extraction Framework of Superconductors related Information from Scientific literature}

\author[1]{Luca Foppiano\thanks{FOPPIANO.Luca@nims.go.jp}}
\author[1]{Thaer Moustafá Dieb\thanks{MOUSTAFADIEB.Thaer@nims.go.jp}}
\author[1]{Akira Suzuki\thanks{SUZUKI.Akira3@nims.go.jp}}
\author[1]{Masashi Ishii\thanks{ISHII.Masashi@nims.go.jp}}
\affil[1]{MaDIS, National Institute for Materials Science}

\date{February 2019}

\begin{document}

\maketitle

\begin{abstract}
Automatic collection of materials information from research papers using natural language processing is highly required for rapid materials development using big data, namely materials informatics (MI). Difficulty of this automatic collection is mainly caused by the variety of expressions in the papers, a device with tolerance to such variety is required to be developed. 
In this paper, we report an ongoing interdisciplinary work to construct the device for automatic collection of superconductor-related information from scientific literature using text mining techniques. We focused on identification of superconducting materials and their key property of critical temperature (Tc), and discussed machine learning (ML) techniques, including annotation strategies to obtain appropriate training data. We introduce a guideline for the annotation together with our several trails of ML on subsequent automatic data collection.

%Research in superconductors is always articulated over two main axes, finding new conditions or discovering new materials (or combination of it) show new or better superconductivity properties. 
%In order to do so, material scientists needs to have rapid access to materials known to be superconductors and their properties without have to examine the thousand of papers related to it. Such data can also be used by further systems to compute generative models 

\end{abstract}

\pagebreak

\tableofcontents

\pagebreak

\section{Introduction}
% What is the problem we are trying to solve? What are the motivation behind this project? 

The goal of this project is to automatically extract domain specific information from scientific articles. Such information are divided into several component, usually names associated to a material together with their properties. This imply not only extraction but also to find a reliable way to cluster or link different properties to the material described. 

In this paper we propose our solution when working on sub-domain of the superconductor material research. 
While each domain has different styles, policies and approaches we still believe a general technique can be partially or entirely reused across domains. In our preliminary assessment we have found several common "problems" between the superconductor and polymer domains, we believe that a more generic approach is still worth the effort. 


\section{Requirements}
%% How research is made and what are the point of improvements

In this section we discuss the superconductor research requirements on two main aspects. First we describe how the research is done, which tasks are performed, and which information and resources are used. Second we discuss more in details the properties and information that domain experts consider important for their daily work. 

% from wikipedia - begin 
Superconductivity is a phenomenon of exactly zero electrical resistance and expulsion of magnetic flux fields occurring in certain materials, called superconductors, when cooled below a characteristic critical temperature. It was discovered by Dutch physicist Heike Kamerlingh Onnes on April 8, 1911, in Leiden. Like ferromagnetism and atomic spectral lines, superconductivity is a quantum mechanical phenomenon.
% end
The superconductor research aims to find new condition on known materials where no resistance is measured (thus it's said "they show superconductivity characteristic"), usually by tuning external conditions like pressure, magnetic fields, temperature or by changing internal structure of a material like having different compositions (called doping percentage) or combining different type of materials.

The goal to obtain the highest temperature possible for which superconductivity characteristics are showing. Ideally room temperature. This is why papers call it "high temperature superconductivity". 

Another aspect of superconductivity research is to discover if some new material (usually a composition) might show superconductivity characteristics. The combination are several: for example by combining small layers of different non-superconductor materials, or mixing superconductors and non superconductors material, or, testing additional composition by changing other characteristics, like pressure. 

To facilitate such tasks, it's imperative to have a database (or knowledge base) where these information are decomposed and presented for each material. 
NIMS database Supercon it's a manually curated database containing about 32k inorganic superconductors definitions, discovered in papers. 
The goal of this project is to find a realiable way to generate such database automatically. 

\subsubsection{Properties of interests}
In the next section we present which information the domain experts in superconductors are interested to be extracted. 

\begin{itemize}
    \item T\textsubscript{c} (Critical Temperature): The temperature recorded for which a specific material shows superconductivity characteristics (zero or lower resistance). Superconductivity is usually not happening sudden, T\textsubscript{c} is sometimes split into two phases T\textsubscript{onset} and T\textsubscript{offset}.
    \item T\textsubscript{onset} (Onset temperature): Temperature where the resistance is still \textgreater 0
    \item T\textsubscript{offset} (Offset temperature): Temperature where the resistance is 0
    \item H\textsubscript{c}: Critical current measured in combination with T\textsubscript{c}.
    \item I\textsubscript{c}: Critical current measured in combination with T\textsubscript{c}.
    \item J\textsubscript{c} (Critical current density): Critical current density measured in combination with T\textsubscript{c} and I\textsubscript{c}.
    \item H\textsubscript{ivr} (Irreversibility field) % need to describe
    \item Cristal structure space group % need to describe
    \item Sample preparation:
    \begin{itemize}
        \item single crystal
        \item poly cristal 
        \item thin film
        \item wire
    \end{itemize}% need to describe
    \item plots or charts
    % here I cannot really write any meaningful sentence, so I what I had in my notes and we add sentences around it later :-) 
    \begin{itemize}
        \item T\textsubscript{c} vs x (with x any special characteristics described in the paper (\% doping, pressure, etc..) 
        \item Magnetisation vs T\textsubscript{c}
    \end{itemize}
\end{itemize}

When researchers describe this characteristics, they don't follow a single approach while sometimes such characteristics are written down in text, they can be presented as plots. 
In our solution problem we have an additional task, then, which is to extract such plots and provide a quick access. A domain expert could quickly evaluate whether the paper is interesting, just by looking at the plots. 

\section{Annotations}
% In this section we are discussing the process of annotation, in particular: 1. how did we find the right balance of annotation, 2.  results in term of IAA, 

Machine Learning bring many advantages in term of accuracy and precision [add ref], tolerance with noise [add references] and flexibility in recognising entities that have never been seen before. 
Unfortunately machine learning systems require training data, and depending of the task they can or cannot be available.

When training data is not available, is a necessity to create it, however the process is simple and requires several cycles of iterative work.

From the practical point of view:  
\begin{enumerate}
    \item it's very expensive in term of time and resources
    \item the time span can be very long before the system starts performing
    \item is a tedious and frustrating work for annotators (usually domain experts feeling overly-skilled, thus less motivated)
    \item it's impossible to be fast and precise 
\end{enumerate}
    
The first important step is always to understand the problem and try to find the simplest approach; find the right balance of accuracy and efficiency. 

\subsubsection{Guidelines}

Annotation guidelines is the set of rules and agreement that are defined for the annotation task. It's a living documentation describing how each entity should be annotated, including special cases. 
Like the training data itself, the process of writing guidelines requires an iterative approach with further improvement for each annotation cycle. 

Annotation guidelines are written as a collaborative work between domain experts and engineers. Before involving the domain experts, however a reasonable approach is to agree a common understanding "internally" among engineers and data scientists that are actively designing the machine learning system. 
This approach bring two advantages: 
\begin{itemize}
    \item assessment and knowledge of the domain, 
    \item it is likely to hit more problems than what imagined by actively performing the task 
    \item early issues help to think more proactively about possible shortcut, solution or additional constraints, and 
    \item we would be better prepared to justify eventual decision to the domain experts
\end{itemize}

The outcome is then discussed and validated with domain experts. 

\subsubsection{Approach for Superconductors}
In this section we describe in detail the first and second iteration and further discussions emerged from our work.

% first iteration
As a first step we, engineers and data scientists, have annotated two sample documents, without any special constraints, to the best to our knowledge and then we have compared the results. 

the goal was to get a first understanding of the papers and exchange our unique view about the task to be solved. 

here the result...


What has emerged were the following points: 
\begin{itemize}
    \item Annotation should be short, and should be split when possible, for example "the Critical Temperature (Tc) should be annotated as <ann1>Critical Temperature</ann2> (<ann1>Tc<ann1>). 
    \item This model should be always be considered as a complementary model to the grobid-quantities masurement recognition, which already provide a general approach to masurement recognition and normalisation. 

    \item We annotated the following items
    \begin{itemize}
        \item <supercon> to identify any superconductor material (for example LaFe03NaCl2, but also "Ba111 serie")
        \item <substitution> to identify substitution of values, for example LaFe\textsubscript{x} might have specified values of x 
        \item Critical temperature should be annotated as <tc>
        \item <propertyValue> define values of any property of a supercondcutor material 
        \item adjective to Tc should not be annotated (e.g. higher Tc, should be considered only Tc)
    \end{itemize}
\end{itemize}





\section{State of the art} 

\section{Method and problems}

\section{Solution}

\section{Conclusion}

\listoffigures

\bibliography{references}
\bibliographystyle{plain}

\end{document}
