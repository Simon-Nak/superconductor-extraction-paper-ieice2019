\documentclass{article}
\usepackage[utf8]{inputenc}
\usepackage{authblk}
\usepackage{tabularx}
\usepackage{url}

\title{Proposal of Automatic Extraction Framework of Superconductors related Information from Scientific literature}

\author[1]{Luca Foppiano\thanks{FOPPIANO.Luca@nims.go.jp}}
\author[1]{Thaer M. Dieb\thanks{MOUSTAFADIEB.Thaer@nims.go.jp}}
\author[1]{Akira Suzuki\thanks{SUZUKI.Akira3@nims.go.jp}}
\author[1]{Masashi Ishii\thanks{ISHII.Masashi@nims.go.jp}}
\affil[1]{Research and Services Division of Materials Data and Integrated System (MaDIS), National Institute for Materials Science (NIMS), 1-2-1 Sengen, Tsukuba, Ibaraki 305-0047, Japan}

% \date{April 2019}

\begin{document}

\maketitle

\begin{abstract}
Automatic collection of materials information from research papers using natural language processing is highly required for rapid materials development using big data, namely materials informatics (MI). Difficulty of this automatic collection is mainly caused by the variety of expressions in the papers, a device with tolerance to such variety is required to be developed. 
In this paper, we report an ongoing interdisciplinary work to construct the device for automatic collection of superconductor-related information from scientific literature using text mining techniques. We focused on identification of superconducting materials and their key property of critical temperature (Tc), and discussed machine learning (ML) techniques, including annotation strategies to obtain appropriate training data. We introduce a guideline for the annotation together with our several trails of ML on subsequent automatic data collection.
\end{abstract}

%% The table of content is there just for organisation purposes, will be removed 
\pagebreak

\tableofcontents

\pagebreak

%Research in superconductors is always articulated over two main axes, finding new conditions or discovering new materials (or combination of it) show new or better superconductivity properties. 
%In order to do so, material scientists needs to have rapid access to materials known to be superconductors and their properties without have to examine the thousand of papers related to it. Such data can also be used by further systems to compute generative models 


\section{Introduction}
% What is the problem we are trying to solve? What are the motivation behind this project? 

Automatic extraction of information from research papers using Natural Language processing is a highly required approach in many domains. In material research, the availability of large quantity of experimental data can give hints and ideas leading to new break-trough materials discoveries. Large availability of scientific papers and the expertise costs for manually generated such data justify the needs of Text and Data Mining automatic approaches.

In this paper we describe the ongoing attempt to design a system aiming to automatically extract superconductor information from scientific literature based on Machine Learning, using Natural Language Processing techniques.

Writing style, variability in experiment and result description are just two of the variables making this task particularly complex. Reason why the best approach to successfully deliver a functioning system is to reduce the complexity to the smallest viable product.
We focus on extracting materials names with doping rates and critical temperature (Tc). We foreseen the use of probabilistic or neural models, our work begin designing the training data schema and define a set of guidelines \cite{article} 
In order to assess the feasibility of our approach, while producing a working prototype in short time, we based our design on an existing open source product: Grobid \cite{GROBID} \cite{lopez2009grobid} a Machine Learning library for extracting information from scholarly documents. We implemented a module specialised in superconductors papers.

As an interdisciplinary collaboration, this project we receive help and collaborate with the material scientists and engineers division. 

%Summary of the next sections - to be discussed
This article is divided as following. In Section \ref{sec:requirements} we describe the domain and the domain expert's requirements. In Section \ref{sec:overview} we will then describe the overall system design and the process of designing the schema for the ML training data. 

\section{The Superconductor research domain}
\label{sec:requirements}
%% How research is made and what are the point of improvements

% from wikipedia - begin 
\textit{Superconductivity is a phenomenon of exactly zero electrical resistance and expulsion of magnetic flux fields occurring in certain materials, called superconductors, when cooled below a characteristic critical temperature. It was discovered by Dutch physicist Heike Kamerlingh Onnes on April 8, 1911, in Leiden. Like ferromagnetism and atomic spectral lines, superconductivity is a quantum mechanical phenomenon.}\footnote{\url{https://en.wikipedia.org/wiki/Superconductivity}}
% end

The research in superconductor materials is articulated toward many different objectives. Discovery of new characteristic of well known materials, under new environment condition, like applied pressure or magnetic field. Combination of known superconductors with non-superconductors may lead to new materials with better characteristics, usually a higher critical temperature. A superconductor scientist accessing a large detailed database with extracted properties from materials could find new leads toward his research, like designing new experiments with different conditions. 

% Should we need to mention supercon? Maybe not 
NIMS database SuperCon\cite{SuperCon} it's a manually curated database containing about 32k inorganic superconductors definitions, discovered in papers. 
The goal of this project is to find a reliable way to generate such database automatically. 

\subsection{Information of interest}
As part of the usual step of assessing the feasibility, and collecting requirements from domain expert, we briefly describe information that domain expert ideally expects to be available in a automatically generated knowledge base for superconductors. 

\begin{itemize}
    \item T\textsubscript{c} (Critical Temperature): The temperature recorded for which a specific material shows superconductivity characteristics (zero or lower resistance). Superconductivity is usually not happening sudden, T\textsubscript{c} is sometimes split into two phases T\textsubscript{onset} and T\textsubscript{offset}.
    \item T\textsubscript{onset} (Onset temperature): Temperature where the resistance is still \textgreater 0
    \item T\textsubscript{offset} (Offset temperature): Temperature where the resistance is 0
    \item H\textsubscript{c}: Critical current measured in combination with T\textsubscript{c}.
    \item I\textsubscript{c}: Critical current measured in combination with T\textsubscript{c}.
    \item J\textsubscript{c} (Critical current density): Critical current density measured in combination with T\textsubscript{c} and I\textsubscript{c}.
    \item H\textsubscript{ivr} (Irreversibly field) % need to describe
    \item Crystal structure space group % need to describe
    \item Sample preparation:
    \begin{itemize}
        \item single crystal
        \item poly crystal 
        \item thin film
        \item wire
    \end{itemize}% need to describe
    \item plots or charts
    % here I cannot really write any meaningful sentence, so I what I had in my notes and we add sentences around it later :-) 
    \begin{itemize}
        \item T\textsubscript{c} vs x (with x any special characteristics described in the paper (\% doping, pressure, etc..) 
        \item Magnetisation vs T\textsubscript{c}
    \end{itemize}
\end{itemize}

When researchers describe this characteristics, they don't follow a single approach while sometimes such characteristics are written down in text, they can be presented as plots. 
In our solution problem we have an additional task, then, which is to extract such plots and provide a quick access. A domain expert could quickly evaluate whether the paper is interesting, just by looking at the plots. 

\section{System design}

\subsection{Overview}
\label{sec:overview}

\subsection{Training data}
% In this section we are discussing the process of annotation, in particular: 1. how did we find the right balance of annotation, 2.  results in term of IAA

Machine Learning bring many advantages in term of accuracy and precision [add ref], tolerance with noise [add references] and flexibility in recognising entities that have never been seen before. 
Unfortunately machine learning systems require training data, and depending of the task they can or cannot be available.

When training data is not available, is a necessity to create it, however the process is simple and requires several cycles of iterative work.

From the practical point of view:  
\begin{enumerate}
    \item it's very expensive in term of time and resources
    \item the time span can be very long before the system starts performing
    \item is a tedious and frustrating work for annotators (usually domain experts feeling overly-skilled, thus less motivated)
    \item it's impossible to be fast and precise 
\end{enumerate}
    
The first important step is always to understand the problem and try to find the simplest approach; find the right balance of accuracy and efficiency. 

% This section should describe how we are doing and which problems we are facing
\section{Results}

\subsection{Annotation principles}

Annotation guidelines is the set of rules and agreement that are defined for the annotation task. It's a living documentation describing how each entity should be annotated, including special cases. 
Like the training data itself, the process of writing guidelines requires an iterative approach with further improvement for each annotation cycle. 

Annotation guidelines are written as a collaborative work between domain experts and engineers. Before involving the domain experts, however a reasonable approach is to agree a common understanding "internally" among engineers and data scientists that are actively designing the machine learning system. 
This approach bring two advantages: 
\begin{itemize}
    \item assessment and knowledge of the domain, 
    \item it is likely to hit more problems than what imagined by actively performing the task 
    \item early issues help to think more proactively about possible shortcut, solution or additional constraints, and 
    \item we would be better prepared to justify eventual decision to the domain experts
\end{itemize}

The outcome is then discussed and validated with domain experts. 

\subsection{Annotating superconductors}
In this section we describe in detail the first and second iteration and further discussions emerged from our work.

% first iteration
As a first step we, engineers and data scientists, have annotated two sample documents, without any special constraints, to the best to our knowledge and then we have compared the results. 

the goal was to get a first understanding of the papers and exchange our unique view about the task to be solved. 

here the result...


What has emerged were the following points: 

\section{Future work}
\section{Conclusions}

\listoffigures

\bibliography{references}
\bibliographystyle{plain}

\end{document}
