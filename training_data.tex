\documentclass{article}
\usepackage[utf8]{inputenc}
\usepackage{authblk}
\usepackage{tabularx}
\usepackage{url}

\title{Annotation Guidelines}

\author[1]{Luca Foppiano\thanks{FOPPIANO.Luca@nims.go.jp}}
\author[1]{Thaer M. Dieb\thanks{MOUSTAFADIEB.Thaer@nims.go.jp}}
\author[1]{Akira Suzuki\thanks{SUZUKI.Akira3@nims.go.jp}}
\author[1]{Masashi Ishii\thanks{ISHII.Masashi@nims.go.jp}}
\affil[1]{Research and Services Division of Materials Data and Integrated System (MaDIS), National Institute for Materials Science (NIMS), 1-2-1 Sengen, Tsukuba, Ibaraki 305-0047, Japan}

% \date{April 2019}

\begin{document}

\maketitle

\section{Tag-set design}
In collaboration with superconductor domain experts, we discussed the relevant information and how they were used in research activities.
As a result, Table \ref{table:summary-entities-superconductor} provides a summarised description of the relevant entities for domain experts.

\begin{table}[h!]
    \centering
    \begin{tabular}{ | m{5em} | m{8cm}| } 
    \hline
        Name & Description  \\ [0.5ex] 
    \hline\hline
        T\textsubscript{c} & Critical Temperature\\ 
    \hline
        T\textsubscript{onset}, T\textsubscript{offset} & Temperature where the resistance tend to zero (offset) to when is really zero (onset)\\ 
    \hline
        H\textsubscript{c1} & Lower Critical field\\ 
    \hline
        H\textsubscript{c2} & Higher Critical field\\ 
    \hline
        I\textsubscript{c} & Critical current\\
    \hline
        J\textsubscript{c} & Critical current density\\ 
    \hline
        H\textsubscript{ivr} & Irreversibly field\\
    \hline
        Crystal structure space group & (TBD) \\
    \hline
        Sample preparation shape & single crystal, poly crystal, thin film or wire. \\
    \hline    
    \end{tabular}
    \caption{Summary of the relevant entities in superconductor research papers}
    \label{table:summary-entities-superconductor}
\end{table}

Depending on the writing style, these information are often presented as tables and plots because they summarise more effectively giving quicker understanding to a human reader. One plot or table can recap several experiments together without in the same space. 
We focus on extracting information from text, with the idea that complementary information can be added on a second stage. 

\section{Training data}
\label{sec:training-data}
% In this section we are discussing the process of annotation, in particular: 
%   1. how did we find the right balance of annotation, 
%   2. results in term of IAA

In Text and Data mining, Machine Learning is currently providing better accuracy and precision, more tolerance to noise and flexibility in recognising entities that have never been seen before. These advantages are not coming for free, because in supervised learning, the system requires a certain amount of examples for training, test and validation. 

% The process of annotation of new training data is very expensive: 
% \begin{enumerate}
%     \item greedy in term of time and resources
%     \item the system might require a lot of data before showing performance accuracy improvements
%     \item is a tedious and frustrating for annotators (usually domain experts,  feeling overly-skilled, thus less motivated)
%     \item throughput and precision are inversely proportional
% \end{enumerate}

In interdisciplinary projects, the process of annotation is always a collaborative work between ML engineers and domain experts. While the former are responsible to deal with the practical complexity of the problem, the latter steer over the content importance and the features requirements. 

After the initial introduction with the domain experts, we, the ML engineers and data scientists, have spent time exploring the domain and attempting to define an annotation schema based on our understanding together with the knowledge of the technologies in our hands. We striven to reach a common understanding and "internally" agreed among us, before requesting the validation from the domain experts. 
In this way we a) asses our knowledge of the domain, b) increase the probabilities to spot problems related to the data by working with it in early stages, c) develop a proactive thinking of possible shortcut or additional constraints and, last but not least, we are ready to justify any additional constraints or decision to domain experts. 

% This section should describe how we are doing and which problems we are facing
We have selected two Open Access papers deposited on Arxiv (Creative Commons) and distributed among us (3 people). We have then pre-annotated them using the current prototype and corrected using 4 labels: <supercon>, <tc>, <propertyValue> and <substitution> to identify respectively superconductor materials, critical temperature expression, values and variables substitution combination. 

We have performed 3 iterative cycles of annotations, at the end of which, after calculating our Inter Annotation Agreement (IAA) we reviewed thoughtfully while updating the annotation guidelines, a "living" document describing how annotate each labels.

In Table \ref{table:summary-iaa} we summarise the IAA in these iterations. Looking at these data, we can see that <substitution> was the more unclear label, as had very low agreement until the 3rd iteration. This is due to the fact that it was appearing in many different form. On the other hand any mention of critical temperature (label <tc>) was more clear (reach 85\% agreement at iteration 2). 

\begin{table}[h!]
    \centering
    \begin{tabular}{ | c | c| c| } 
    \hline
        Iteration \# & IAA & IAA by label  \\ [0.5ex] 
    \hline\hline
        1  & 0.45
        &\begin{tabular}{  c | c  } 
            supercon & 0.45\\ 
            tc & 0.56\\
            propertyValue & 0.50\\
            substitution & 0.21\\
        \end{tabular}    
        \\ 
    \hline
        2 & 0.65
        &\begin{tabular}{  c |  c  } 
            supercon & 0.75\\ 
            tc & 0.85\\
            propertyValue & 0.85\\
            substitution & 0.39 \\
        \end{tabular}          
        \\ 
    \hline
        3 & 0.89
        & \begin{tabular}{  c | c  } 
            supercon & 0.89\\ 
            tc & 0.91\\
            propertyValue & 0.88\\
            substitution & 0.94\\
        \end{tabular}       
        
        \\ 
    \hline
    \end{tabular}
    \caption{Summary of the IAA for each of the three cycles of annotation. Together with the average annotation agreement, we publish the agreement by label.}
    \label{table:summary-iaa}
\end{table}

\section{Conclusion}

% IAA
We have analysed the process and the evolution of the agreement during our process and how this can be used to understand when certain constraints or definition are to be clarified further. 
On the creation of annotation we should improve the process of creating training data, especially when the domain expert will be actively involved in task. 

\end{document}